\section{ggplot2 - Part 3}
\subsection{statistics in ggplot2}
\begin{frame}\frametitle{Statistics}
  \begin{itemize}
  \item useful to transform your data before plotting, and that's what statistical transformations do
  \item name convention \texttt{stat\_specification}
  \item every statistic has a default geometry, e.g. \texttt{stat\_bin()} uses \texttt{geom\_bar()}
  \end{itemize}
\end{frame}


\begin{frame}[allowframebreaks,fragile]\frametitle{\texttt{stat\_bin()}}
  \begin{itemize}
  \item bins (and counts) (grouped) data 
  \end{itemize}
\small
\begin{verbatim}
> ggplot(data,aes(x=EC1)) +
+     geom_bar() +
+     stat_bin(geom="text", aes(label=..count..),
+              colour="red", size=14, vjust=1)
ymax not defined: adjusting position using y instead
\end{verbatim}
\begin{center}
  \includegraphics[width=6.5cm]{statbin.png}
\end{center}
\end{frame}


\begin{frame}[allowframebreaks,fragile]\frametitle{\texttt{stat\_density()} \& \_function()}
\texttt{stat\_density()}
  \begin{itemize}
  \item computes and plots density estimates
  \item allows different kernels and bandwidths
  \item default geometry: \texttt{geom\_density()}
  \end{itemize}
\texttt{stat\_function()}
  \begin{itemize}
  \item plots functions, needs only one aesthetic: \texttt{y}
  \item default geometry: \texttt{geom\_path()}
  \end{itemize}
\small
\begin{verbatim}
> ggplot(data,aes(x=log(TTime))) +
+     geom_density() +
+     stat_function(fun=dnorm,
+                   args = list(mean=mean(log(data$TTime)),
+                               sd=sd(log(data$TTime)))
+                   )
\end{verbatim}
\begin{center}
  \includegraphics[width=10cm]{statdens.png}
\end{center}
\end{frame}


\begin{frame}[allowframebreaks,fragile]\frametitle{\texttt{stat\_summary()}}
  \begin{itemize}
  \item summarize y values at every unique x value
  \item default geometry: \texttt{pointrange}
  \end{itemize}
\small
\begin{verbatim}
> ggplot(data,aes(x=EC1,y=Duration)) +
+     stat_summary(fun.data="mean_se",geom = "pointrange")
\end{verbatim}
\begin{center}
  \includegraphics[width=10cm]{sum1.png}
\end{center}
\end{frame}


\begin{frame}[allowframebreaks,fragile]\frametitle{\texttt{stat\_summary()}}
  \begin{itemize}
  \item one can also create custom summary functions or stats
  \end{itemize}
\footnotesize
\begin{verbatim}
trimed.mean <- function(x){
+     data.frame(y=mean(x,na.rm = T,trim = 0.1))
+ }
> stat_meanlabel <- function(angle=0,vjust=0.5,hjust=0,...){
+     stat_summary(fun.y="mean",
+                  geom="text",
+                  aes(label=round(..y..)),
+                  hjust=hjust,
+                  vjust=vjust,
+                  angle=angle, ...)}
> ggplot(data,aes(x=EC1,y=Duration)) +
+     stat_summary(fun.data="mean_se",geom = "pointrange") +
+     stat_summary(fun.data="trimed.mean",geom = "point", colour="blue") +
+     stat_summary(fun.y="median",geom = "point", colour="green") +
+     stat_meanlabel(angle = 90,vjust = 0 )
\end{verbatim}
\begin{center}
  \includegraphics[width=10cm]{sum2.png}
\end{center}
\end{frame}

\subsection{annotations}
\begin{frame}[allowframebreaks,fragile]\frametitle{\texttt{annotation()}}
  \begin{itemize}
  \item creates an annotation layer (a geometry)
  \item properties of the geoms are not mapped from variables but
  \item information given as vector
  \item one can add simple things as labels, segments or points or
  \item more complex elements using the gridExtra package
  \end{itemize}\tiny
\begin{verbatim}
> ggplot(data,aes(x=EC1,y=Duration)) +
+     stat_summary(fun.data="mean_se",geom = "pointrange") +
+     annotate(geom="text",x=3,y=15000,label="here we are") +
+     annotate(geom="segment",x = 2.5,xend = 3.5, 
+              y = 15000,yend=16000,colour="red") +
+     annotate(geom="rect",xmin = 4.8,xmax = 6.2, 
+              ymin = 19100,ymax=20100,fill="red",alpha=0.3)
\end{verbatim}
\begin{center}
  \includegraphics[width=10cm]{annot.png}
\end{center}
\begin{verbatim}
> my.table <- tableGrob(as.data.frame(table(data$EC1)))
> ggplot(data,aes(x=EC1,y=Duration)) +
+     stat_summary(fun.data="mean_se",geom = "pointrange") +
+     annotation_custom(xmin = 1, xmax = 3, 
+                       ymin = 18000, ymax = 20000,
+                       grob = my.table) +
+     ylim(13000,20000)  
\end{verbatim}
\begin{center}
  \includegraphics[width=10cm]{annot1.png}
\end{center}
\begin{verbatim}
> my.subplot <- ggplotGrob(
+     p1 +
+         theme(plot.background=element_rect(fill="transparent",
+                                            colour="transparent"),
+               panel.background=element_rect(fill="transparent"),
+               panel.grid=element_blank()))
> ggplot(data,aes(x=EC1,y=Duration)) +
+     stat_summary(fun.data="mean_se",geom = "pointrange") +
+     annotation_custom(xmin = 1, xmax = 3, 
+                       ymin = 18000, ymax = 20000,
+                       grob = my.table) +
+     annotation_custom(xmin = 4, xmax = 6, 
+                       ymin = 13000, ymax = 18000,
+                       grob = my.subplot) +
+     ylim(13000,20000)  
\end{verbatim}
\begin{center}
  \includegraphics[width=10cm]{annot2.png}
\end{center}
\end{frame}

\begin{frame}\frametitle{ggplot Exercises}
  \begin{enumerate}
  \item use \texttt{stat\_summary()} to plot the mean inclusive the 95 percent confidence interval per time (testid) and use \texttt{stat\_bin()} to add the number of observations, create one facet per subject
  \item add a new column to the data data frame containing a 1 if Stim.Type equals hit and 0 otherwise, now use \texttt{stat\_summary()} to plot a line (time on the x-axis and the percentage of correct hits on the y-axis. Hint: you have to transform testid into a numeric variable.
  \item now colour the background of the time points test1 and test2 different from the training time. use \texttt{annotate()} and label the x-axis correct
  \item add two arrows indicating the minimal and maximal percentage of correct answers, use segment in combination with \texttt{arrow=arrow()} (load the gridExtra package)
  \end{enumerate}
\end{frame}


\begin{frame}[allowframebreaks,fragile]\frametitle{ggplot Exercises}
  \begin{enumerate}
  \item use \texttt{stat\_summary()} to plot the mean inclusive the 95 percent confidence interval per time (testid) and use \texttt{stat\_bin()} to add the number of observations, create one facet per subject
\begin{verbatim}
> ggplot(data,aes(x=testid,y=TTime)) +
+     stat_summary(fun.data="mean_se",mult=1.96,geom="pointrange") +
+     stat_bin(y=0,aes(label=..count..),geom="text") +
+     scale_y_continuous(limits=c(0,75000)) +
+     facet_wrap(~Subject)
\end{verbatim}
\begin{center}
  \includegraphics[width=8cm]{statsumexer.png}
\end{center}
  \item add a new column to the data data frame containing a 1 if Stim.Type equals hit and 0 otherwise, now use \texttt{stat\_summary()} to plot a line (time on the x-axis and the percentage of correct hits on the y-axis. Hint: you have to transform testid into a numeric variable.
\begin{verbatim}
> data$correct <- as.numeric(data$Stim.Type=="hit")
> ggplot(data,aes(x=as.numeric(testid),y=correct)) +
+     stat_summary(fun.y="mean",geom="line")
\end{verbatim}
\begin{center}
  \includegraphics[width=8cm]{statsumexer2.png}
\end{center}
  \item now colour the background of the time points test1 and test2 different from the training time. use \texttt{annotate()} and label the x-axis correct
  \item add two arrows indicating the minimal and maximal percentage of correct answers, use segment in combination with \texttt{arrow=arrow()} (load the gridExtra package)
\begin{verbatim}
> require(gridExtra)
> ggplot(data,aes(x=as.numeric(testid),y=correct)) +
+     stat_summary(fun.y="mean",geom="line") +
+     annotate("rect",xmin = 0.5,xmax = 1.5,
+              ymin = -Inf, ymax = Inf,
+              fill = "red", alpha = 0.5) +
+     annotate("rect",xmin = 9.5,xmax = 10.5,
+              ymin = -Inf, ymax = Inf,
+              fill = "red", alpha = 0.5) +
+     annotate("segment",
+              x=9,xend=9,
+              y=0.6,yend=0.55,size=3,arrow=arrow()) +
+     annotate("segment",
+              x=1,xend=1,
+              y=0.7,yend=0.75,size=3,arrow=arrow())
\end{verbatim}
\begin{center}
  \includegraphics[width=6.5cm]{exerannot.png}
\end{center}
  \end{enumerate}
\end{frame}
